\section{Introduction}

Virtual reality is a new computing environment, while the primary use is in games and videos, it can be used in many application domains, from education, architectural reviews~\cite{guerreiro2014beyond}, healthcare, large group chats ~\cite{mcnerney1999system}, and data visualization~\cite{abbott2011empire}, exploration of space and other dangerous locations, scientific visualization, manufacturing, journalism, traveling, architecture, shopping.
As the VR equipment and content improves, we can imagine that VR system can provide a new learning and working environment, much like how the PC has transformed our daily life.

Today's commercial virtual reality systems have relatively primitive input devices.
The most advanced controller available are the hand-tracked controllers provided by HTC Vive.
They enable users to interact with the objects in the virtual space, and it includes a trigger and a trackpad style click wheel.
The Oculus Rift currently uses a game controller, and the Samsung Gear has a couple of buttons on the headset.

This paper studies a very basic form of input in virtual reality environments: text.
While users do not need to type much in immersive virtual reality environments we find today, text will become more useful as VR matures.
At the basic level, the application may need you to input parameters: such as your account name and password, or the name of the city that you want to visit.
Second, text is useful for searching and navigating content.
For example, you may wish to open a game recommended from a friend - you may know it by its name, or your friend has sent you a textual link.
You may not know the name, and you may not need to search.
Or you may wish to type in keywords to search for a friend in a social app, a product in a shopping app, or a patient’s record in a health app.
Third, you may want to take notes as you navigate the content for yourself or to communicate with your friends.
Finally, you may want to stay in touch with the actual reality while immersed in the virtual reality.
You may wish to receive messages and provide prompt replies without leaving the environment.

Many of us have been frustrated with text input on smart TVs. Imagine the common scenario where we wish to show a YouTube video to a friend on the TV in a living room.
The easiest is to use your phone, find the content, and use screencast to show it on the TV.
The keyboard entry on a TV is so clumsy that we would rather not search for the content in YouTube on TV and we definitely do not wish to type in the long YouTube page that we have open on our phone.
Virtual reality is even harder because of the lack of proprioception.
Today, our solution is to remove our head gear and to find the content on the PC or on the smart phone.
Having a fast and convenient way of entering text is fundamentally desirable.

So far, the alternatives have been through gesture, motion controllers, Leap Motion or tangibles, handheld controllers, or relying on a subset of keyboard and mouse commands~\cite{billinghurst1999collaborative}.
All these existing solutions, while potentially good for simple tasks, aren't adequate for tasks that requires a greater bandwidth of input~\cite{McGill:2015:DRO:2702123.2702382}.

