\section{Future Work}
\subsection{Editing of text}
Most skilled manual activities involve two hands playing different roles ~\cite{guiard1987asymmetric, Casalta:1999:ETI:632716.632862}.
The two hands cooperate with one another as if they were assembled in series, thereby forming a kinematic chain.

More work has to be done, but for some tasks, such as, such as navigating and editing a document, it could be possible that This theory fits ell with the SwipeVR input controller.

Whilst the dominant hand is entering words, the non-dominant hand can be navigating the dominant to set the position in the document.
This could be done in one fluid task without having to clutch into a different input mode.

This is in  comparison to the typical setup whereby the user remove there dominant hand from the keyboard to reach for the mouse, navigate to the correct place in the document, then place there dominant and back on the keyboard. 
Meanwhile the non-dominant remains still

\subsection{Flexion and Extension of the Wrist}
The flexion and extension of the wrist~\cite{sarrafian1977study} is a high bandwidth bus, ~\cite{TBD}, almost as much as a finger, for information entry from the physical world into the virtual.
SwipeVR does not use this channel.
For example, the user could use the trigger button along with the wrist to select and move text around the document.
Again, while the dominant hand does the selecting, the non-dominant hand can reposition document or virtual world.

